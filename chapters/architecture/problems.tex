
%|  Name  | TODO | ONGOING | DONE |
%|--------|------|---------|------|
%| Dana   | x    |         |      |
%| Gerd   |      |   x     |      |
%| Glenn  | x    |         |      |
%| Jordan | x    |         |      |
%| Luke   | x    |         |      |
%| Matt   | x    |         |      |
%| Neil   | x    |         |      |
%| Scot   | x    |         |      |


%\todo[inline]{Owner: Neil -- Priority: Medium -- Effort: M -- Completion: 0\%}

The purpose of this section is to collect and document known \textit{architectural} issues in the HDF5 library. For now, this is more of a placeholder, but not because there aren't any known issues. It's hard to figure out where to start. So, watch this space!

Perhaps a good starting point would be to approach this topic from the perspective of flexibility~\cite{hanson2021} or extensibility~\cite{Woods2018}. How has the HDF5 library architecture changed over 25 years? What drove those changes? Were they graceful or at least successful?

\begin{quote}
    ``Our goal in this book is to investigate how to construct computational systems so that they can be easily adapted to changing requirements. One should not have to modify a working program. One should be able to add to it to implement new functionality or to adjust old functions for new requirements. We call this \textit{additive programming}. We explore techniques to add functionality to an existing program without breaking it. Our techniques do not guarantee that the additions are correct: the additions must themselves be debugged; but they should not damage existing functionality accidentally."~\cite{hanson2021}
\end{quote}

What are the ``multiplicative" elements in the HDF5 library architecture?

Looking at the partial chronology in Appendix~\ref{app:chronology}, how many of those features have made the HDF5 library a better product?

What is the scale of accumulated technical debt in the code base, and what fraction is due to poor architecture?

% \section{C-API Issues}

% \begin{itemize}
%     \item Inadequate error reporting mechanisms
%     \begin{itemize}
%         \item Usually just -1, \texttt{FAIL}, or \texttt{H5I\_INVALID\_HID} (no fine-grained error codes)
%         \item Parsing error stacks is complicated and doesn't work well due to inconsistent major/minor error values
%     \end{itemize}
% \end{itemize}

% \section{VOL Issues}

% \begin{itemize}
%     \item Limited VOL capability introspection
% \end{itemize}

% \section{Non-VOL (Infrastructure) Issues}

% \begin{itemize}
%     \item Entanglement with the native VOL
%     \item Multi-threading
% \end{itemize}

% \section{Native VOL Issues}

% \begin{itemize}
%     \item Obscure defaults/lack of configurability
%     \item Poor encapsulation (generally only at the package level)
%     \item Lots of small I/O operations maps poorly to high-performance I/O systems
% \end{itemize}