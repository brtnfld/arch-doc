\todo[inline]{Owner: Dana -- Priority: High -- Effort: M -- Completion: 5\%}

\section{Operating Systems}

\subsection{Windows}
\begin{itemize}
    \item SWMR should work on Windows with local file systems (NOT SMB shares), though testing is not a robust as on MacOS/Linux
    \item Parallel HDF5 is poorly tested on Windows
    \item The direct VFD is not supported as it uses the POSIX \texttt{O\_DIRECT} flag
    \item Thread-safety uses win32 threads
    \item Some API calls use \texttt{off\_t}, which is 32-bits on Windows (we use \texttt{\_\_int64} internally)
    \item Unicode file paths are not well-supported
\end{itemize}

\section{Build Systems}

All features should be supported in both the Autotools and CMake.

\section{Feature Compatibility}

Only incompatibilities are noted (an empty cell implies compatibility).

\begin{minipage}{\textwidth}
  \begin{center}
    \begin{tabular}{|| c | c | c | c ||}
      \hline
      Feature & Serial & SWMR (serial) & Parallel \\
      \hline
      Free-space tracking & ? & ? & ?\\
      \hline
      Paged Allocation & ? & ? & ? \\
      \hline

      Virtual Dataset (VDS) Layout & ? & ? & ? \\
      \hline
      Variable-length datatypes & ? & ? & ? \\
      \hline
    \end{tabular}
  \end{center}
\end{minipage}

\section{Other Miscellaneous Limitations}
\begin{itemize}
    \item The current implementation of virtual datasets doesn't support point selections
    \item The direct VFD requires \texttt{open(2)} to take the \texttt{O\_DIRECT} flag
\end{itemize}

%\begin{minipage}{\textwidth}
%  \begin{center}
%    \begin{tabular}{|| c | c | c | c | c | c ||}
%      \hline
%      & SWMR & Paged Allocation & Free-space tracking & VDS & Non-fixed-size datatypes \\
%      \hline
%      SWMR & S & S? & S? & S & $\times$ \\
%      \hline
%      Paged Allocation & & P\footnote{No page buffering?} & P? & P? & S \\
%      \hline
%      Free-space tracking & & & P? & P? & S \\
%      \hline
%      VDS & & & & P & S \\
%      \hline
%      Non-fixed-size datatypes & & & & &  S \\
%      \hline
%    \end{tabular}
%  \end{center}
%\end{minipage}

%\paragraph{Symbology}
%\todo[inline]{GH: Is this a good way of breaking it down?}
%\begin{itemize}
%    \item S - sequential only
%    \item P - MPI parallel \& sequential
%    \item $\times$ - unsupported
%\end{itemize}

%\paragraph{Other}
%\begin{itemize}
%    \item The current implementation of virtual datasets doesn't support point selections
%\end{itemize}

% Other row/column candidates
% - VDS
% - Paged allocation/page buffering
% - ???

% Wrappers/High-level*
%Parallel vs. \\ Serial & Thread-Safe vs. \\ Not Thread-Safe & Windows/MacOS/ \\ Linux/Misc. POSIX \\
% https://stackoverflow.com/questions/2888817/footnotes-for-tables-in-latex for footnotes in table
% SWMR doesn't support variable-length and region reference datatypes

% Collective metadata writes aren't supported with page buffering, separate table probably(?)
% Page buffering doesn't support parallel HDF5

% Free space mangement

% ROS3 VFD -> 
% Lots of missing documentation

% Direct VFD

% Core, Direct, Onion, Family, Log, MPIO, Multi, Sec2, Splitter, Mirror, etc VFDs? 

% Metadata cache?

% 

\paragraph{VFD stackability and features}
\begin{center}
    \begin{tabular}{|| c | c ||}
      \hline
      \rotatebox{90}{Pass-Through} & family, mirror, multi/split, onion, splitter, subfiling \\
      \hline
      \rotatebox{90}{Terminal} & core, direct, HDFS, Hermes, log, MPI-I/O, sec2, ros3, stdio\\
      \hline
    \end{tabular}
  \end{center}

NOTE: The onion and subfiling VFDs are currently hard-coded to use the sec2 VFD.

\paragraph{VOL connector stackability}